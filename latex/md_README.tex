\#\+D\+A\+TA DA E\+N\+T\+R\+E\+GA\+: 17/08/2017 \section*{A\+L\+U\+NA\+: R\+O\+S\+A\+N\+G\+E\+LA D A\+V\+I\+L\+LA V\+I\+C\+E\+N\+TE DE O\+L\+I\+V\+E\+I\+RA W\+I\+T\+TE}

L\+A\+B\+O\+R\+A\+T\+O\+R\+IO 1 -\/ LP 1


\begin{DoxyItemize}
\item Q\+U\+E\+S\+T\+AO 01\+: Para compilá-\/la ,execute o comando \char`\"{}make questao01\char`\"{} com o terminal setado na pasta \char`\"{}\+L\+A\+B\+\_\+\+I\char`\"{}.
\item Q\+U\+E\+S\+T\+AO 02\+: Para compilá-\/la ,execute o comando \char`\"{}make questao02\char`\"{} com o terminal setado na pasta \char`\"{}\+L\+A\+B\+\_\+\+I\char`\"{}.$\ast$A resposta com a utilização do Gprof está no diretório \char`\"{}\+Questao02-\/gprof\char`\"{}.
\end{DoxyItemize}

\begin{DoxyVerb}exemplo:
    ~/LAB-I$ make questao01 
    ~/LAB-I$ make questao01
\end{DoxyVerb}



\begin{DoxyItemize}
\item Q\+U\+E\+S\+T\+AO 03\+: No diretório \char`\"{}./\+Questao3-\/\+G\+D\+B\char`\"{} há informações sobre a questão 03.
\end{DoxyItemize}

C\+O\+M\+E\+N\+TÁ\+R\+I\+OS\+: Este laboratório foi o meu primeiro contato com as ferramentas github,doxygen,gprof e makefile. No início pareceu-\/me um pouco complicado pois não estava acostumada a utilizá-\/los. Posteriormente, com a prática vinda do laboratório,notei como estas ferramentas facilitam a vida de quem programa ,sendo útil e ágil em um projeto de pequeno porte como este e, principalmente, em projetos maiores. Minha maior dificuldado foi entender o Makefile como uma \char`\"{}linguagem\char`\"{} e não sim-\/ plesmente copiar sem a percepção do que eu estava fazendo ou alterando. 